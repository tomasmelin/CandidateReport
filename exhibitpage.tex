\documentclass [cropmarks, frame, english, master]{idaexhibitpage}
\usepackage [latin1]{inputenc}
\isbn {-}
\author {Tomas Melin and Tomas Vidhall }
\thesisnumber {THESISNUMBER}
\titleswedish {Namecoin som autentisering f\~A\IeC {\P  }r asymmetrisk kryptering}
\titleenglish {Namecoin as authentication for public-key cryptography}
\degreetype {teknologie}
\degreesubject {Engineering}
\degreesubjectswedish {DEGREESUBJECTSWEDISH}
\presentationplace {PRESENTATIONPLACE}
\presentationhouse {PRESENTATIONHOUSE}
\presentationhus {PRESENTATIONHUS}
\presentationdate {PRESENTATIONDATE}
\presentationdatum {PRESENTATIONDATUM}
\presentationtime {PRESENTATIONTIME}
\keywords {KEYWORDS}
\nyckelord {NYCKELORD}
\facultyexaminername {OPPONENT NAME}
\facultyexaminertitle {OPPONENT TITEL}
\facultyexamineraddress {OPPONENT ADDRESS}
\thesisurl {http://XXX}
\supportedby {SUPPORTEDBY}
\newcommand {\issn }{-}
\begin {document}
\exhibitpagebeforeabstract 
 \S  \ Public-key cryptography is a subject that is very important to everyone who wants confidentiality and privacy in networks. It is important to understand how public-key cryptography systems work and what flaws they have. In the first part of this report we describe some of the most common encryption schemes and key agreements. We carefully investigate their flaws, if they are broken and what threats have dire consequences. \par \noindent We found that the biggest issue is authentication and we present current solutions to the problem. The current solutions are flawed because they rely too much on trusting different entities. It is only required that one trusted entity becomes malicious for the entire authentication system to be compromised. Because of this we propose an alternative system in the second part, Namecoin. A risk analysis in form of an attack tree is performed on the Namecoin system, where we describe how the attacks are executed and what you can do to prevent them. We present different threats against the system and we describe how dire the consequences are and the probability of their execution. Since Namecoin is an implementation of the block chain algorithm we have also explained how the block chain works in detail. \par \noindent We present why we think that Namecoin is a system that should replace the currently used certificate authority system. The certificate authority system is flawed because it is centralized and dependant on that no authority makes any mistakes. The Namecoin system does not become compromised unless more than $50 \tmspace  +\thinmuskip {.1667em} \%$ of the hashrate in the system is used with malicious intent. We have concluded that the biggest threats against Namecoin have such a low probability that they can be neglected. 
\exhibitpageafterabstract 
\end {document}
