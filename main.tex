\documentclass{acm_proc_article-sp}
\usepackage[utf8]{inputenc}
\usepackage{amssymb}
\usepackage{amsmath}
\usepackage{amsfonts}
\usepackage{mathtools}
%\hyphenation{authentication}
%\hyphenation{cryptography}
%\hyphenation{symmetric}
%\hyphenation{asymmetric}
%\hyphenation{vulnerable}
%\hyphenation{changed}

\begin{document}

%Vilka hot finns mot krypteringsmetoder som använder sig av asymmetrisk kryptering?
\title{What vulnerabilities and threats exists for public-key cryptography?}
\numberofauthors{2}
\author{
\alignauthor
Tomas Melin\\
        \affaddr{Linköping University}\\
        \affaddr{Linköping, Sweden}\\
        \email{tomme578@student.liu.se}
\alignauthor
Tomas Vidhall\\
        \affaddr{Linköping University}\\
        \affaddr{Linköping, Sweden}\\
        \email{tomvi780@student.liu.se}
}
\date{April 12 2014}
% General Comments:
%   
%

\maketitle
%-------------------------------------------------------------
%   ABSTRACT
%-------------------------------------------------------------
\begin{abstract}
%% SKRIV ABSTRAKT
[Awesome Text]
\end{abstract}

\keywords{public key cryptography}

%-------------------------------------------------------------
%   INTRODUCTION
%-------------------------------------------------------------
\section{Introduction}
% Förklaring vad 'public-key cryptography' är.
The public-key cryptography, or asymmetric encryption as it is also referred to, is a procedure with the main objective to ensure privacy or confidentiality. The procedure uses two keys, public and private key. Each person has a public key that is used to encrypt messages and can be shared in any possible way. Each user also has a private key which is only known to that specific user. If a user receives a message that is encrypted with his public key he can decrypt it with his private key and no one else can decrypt it. \\
This can be explained further using mathematical notations. An entity has a public key $e$ and a corresponding private key $d$. For a system to be secure, it has to be fairly impossible to compute $d$, given $e$. The public key defines an encryption transformation $E_e$ and the private key defines the associated decryption transformation $D_d$. This means that if an entity $B$ wants to send a message to $A$, it has to retrieve a copy of $A$'s public key $e$ and use $E_e$ to obtain the cipher text $c=E_e(m)$, where $m$ is the message in plain text, and send $c$ to $A$. Once $A$ has received the cipher text $c$, $A$ uses $D_d$ with the private key $d$ to decrypt the cipher text into the original message in plain text $m=D_d(c)$. \cite{handcrypt}\\
By the above description it is clear that using public-key encryption on its own, does not provide data origin authentication, due to the fact that $A$'s encryption transformation is public information. Data origin authentication can be described as a  mechanism where the data is authenticated. \cite{handcrypt} \pagebreak

%Asymmetrisk kryptering är en metod för att kommunicera säkert. Detta fungerar genom att man har två nycklar, en publik och en privat. 
%Den publika nyckeln är känd för alla men den privata måste vara hemlig. Alla kan kryptera meddelanden med en publik nyckel men de kan bara dekrypteras med hjälp av den privata nyckeln. Denna metod används idag inom bland annat RSA-kryptering. Problem som kan uppstå med denna metod är autentisering av publika nycklar. 

\subsection{Motivation}
The concept with private and public keys is very fascinating to us and we also know that it is very hard to read messages which have been encrypted with a foreign public key. This makes us motivated to study public-key encryption more carefully and investigate the subject thoroughly. We plan to find the advantages and the disadvantages of using public-key encryption and how it is possible to take advantage of its flaws, in order to break the implementation. 

% Vi tycker att konceptet med privata och publika nycklar är väldigt snyggt. Vi vet också att det är “omöjligt” att läsa meddelanden som krypterats med någon annans publika nyckel. Det vi undrar är vilka nackdelar som finns och hur man kan utnyttja dessa.  

\subsection{Purpose}
We want to extend our knowledge in the subject of communicating over a unsafe network using public-key encryption. We also would like to investigate how safe public-key encryption is and what threats that has been a vulnerability, and what threats still exists. 

% Vi vill få reda på mer om hur man kan kryptera data på ett osäkert nätverk genom att använda asymmetrisk kryptering. Vi vill också ta reda på hur säker denna metod är och om vissa hot fortfarande finns kvar. 

\subsection{Questions of Interest}
\begin{itemize} % SE ÖVER, LÄGG TILL
    \item How does public-key cryptography work? %Hur fungerar asymmetrisk kryptering?
    % Förklara tillvägagångssättet (med formler).
    \item How has public-key cryptography implementations\\ changed through history? %Hur har asymmetrisk kryptering historiskt sett förändrats då olika implementationer har knäckts?
    \begin{itemize}
        \item Has it been affected by cracked implementations?
    \end{itemize}
    \item What attacks are public-key cryptographys vulnerable to? %Vilka sorts attacker är asymmetriska krypteringsmetoder sårbar mot?
    \begin{itemize} 
        \item What weaknesses can a Man in the Middle-attack take advantage of? %Vad kan man göra som ‘Man in the Middle’?
        \item Is it possible to prevent these attacks through authentication? %Kan man motverka dessa attacker med autentisering?
    \end{itemize}
    \item How does the Diffie-Hellman key exchange work? %Hur fungerar Diffie-Hellmans metod för nyckelutbyte?   
    \begin{itemize}
        \item What attacks is the Diffie-Hellman public-key exchange vulnerable to? %Vilka sorters attacker är Diffie-Hellman sårbar mot?
        \item What is the benefit of using symmetric cryptography compared to public-key cryptography? %Varför vill man kryptera symmetriskt när man redan kan kommunicera med asymmetrisk kryptering? 
    \end{itemize}
\end{itemize}

\subsection{Boundaries}
This report will not focus on the mathematical aspects of encryption and its algorithms, but will instead focus on technical vulnerabilities and  solutions for these. 

%-------------------------------------------------------------
%   THEORY
%-------------------------------------------------------------
\section{Theory}
%OLIKA SUBSECTIONS MED ATTACKS AGAINST IMPLEMENTAION & CRYPTOGRAPHY.
%Exempelifiera metoder som använder public-key cryptography. 
There have been several public-key encryption schemes throughout the history but the most significant ones are RSA, Rabin, ElGamal, McEliece, Merkle-Hellman knapsack, Probabilistic public-key encryption. The procedure which these encryption schemes use to generate their key is different from scheme to scheme but we will show how a number of those generate their key.
\subsection{Public-key encryption schemes}
%\begin{description}
%   RSA
    \subsubsection{RSA public-key encryption}
    which is short for R. Rivest, A. Shamir, and L. Adleman and is the most widely used public-key cryptographic system. \cite{handcrypt} RSA's security is based on the intractability of the integer factorization problem, that is, given a positive integer $n$, your task is to find its prime factorization. 
    \begin{equation*} % INTEGER FACTORIZATION PROBLEM
        n=p_{1}^{e_1}p_{2}^{e_2} \cdot \cdot \cdot p_{k}^{e_k}
    \end{equation*}
    Where $p_i$ are distinct primes and each $e_i \leq 1$. \cite{handcrypt} The key generation for RSA public-key encryption procedure containing five steps. Each entity is required to have RSA public key and a corresponding private key. 
    \begin{enumerate} % RSA KEY GENERATION
        \item Generate two large random (and distinct) primes $p$ and $q$, each roughly the same size. 
        \item Compute $n=pq$ and $\phi = (p-1)(q-1)$
        \item Select a random integer $e$, $1 < e < \phi$, such that $gcd(e,\phi)=1$, where $gcd(x,y)$ is the greatest common divisor for $x$ and $y$. 
        \item Use the extended Euclidean algorithm to compute the unique integer $d$, $1<d<\phi$, such that $ed \equiv 1 (\bmod \phi)$
        \item The entity's public key is $(n,e)$ and the private key is $d$\hfill \cite{handcrypt}
    \end{enumerate}
    The algorithm for RSA public-key encryption has two phases, encryption and decryption, as described in \cite{handcrypt} on p.286.
    \begin{description}
            \item[Encryption] In this case, $B$ encrypts a message $m$ for $A$, which A decrypts.
            \begin{enumerate}
                \item Obtain A's authentic public key $(n,e)$.
                \item Represent the message as an integer $m$ in the interval $[0,n-1]$.
                \item Compute $c=m^e \bmod n$, using the extended Euclidean algorithm.
                \item Send the cipher text $c$ to $A$. 
            \end{enumerate}
            
            \item[Decryption] To recover plain text $m$ from $c$, $A$ should do the following: 
            \begin{enumerate}
                \item Use the private key $d$ to recover $m=c^d \, \bmod n$. 
            \end{enumerate}
    \end{description}
    
    %%FLYTTA TILL DISCUSSION ELLER NÅT LIKNANDE??? :D
    This version of RSA is what is called the textbook version of RSA, i.e. the version that is taught but is not actually used. The difference between this and the version of RSA that is being used in the real world is that the non-textbook version uses what is called \emph{padding}. Padding can be explained as to insert a structured randomized sum of characters into the message $m$, which is to be encrypted. This is done in order to separate e.g. two equal messages ($m_1$ and $m_2$) and their corresponding cipher texts ($c_1$ and $c_2$) from having similar cipher texts, thus avoiding that a potential attacker can distinguish which cipher text represents which plain text message. 
    
%   RABIN
    \subsubsection{Rabin public-key encryption}
     is the first example of a provably secure public-key encryption scheme. Many cryptographic systems, for example RSA, are believed to be as hard to break as the mathematical problems they build on, but few can really prove it. Rabin is one of those where it is proven to be as secure as the related mathematical problem, the integer factorization problem. Rabin works just as RSA except for differences in key-generation. Each entity $A$ is required to have both a public and a private key and they are generated according to the following steps.
    \begin{enumerate}
        \item Generate two large, random, different prime numbers $p$ and $q$, each roughly the same size.
        \item Compute $n = pq$.
        \item $A$'s public key is $n$ and $A$'s private key is ($p$,$q$).
    \end{enumerate}
    
    \begin{description}
    \item[Case] $B$ encrypts a message $m$ for $A$ which $A$ then decrypts.
    \item[Encryption] What $B$ should do.
    \begin{enumerate}
        \item Obtain $A$'s public key n.
        \item Represent the message as an integer m in the range \{0,1,...,n-1\}.
        \item Compute $c$ = $m^2 \bmod n$.
        \item Send $c$ to $A$.
    \end{enumerate}
    \item[Decryption] What $A$ should do.
    \begin{enumerate}
        \item Use the algorithm to find square roots modulo $n$ given its prime factors $p$ and $q$. This gives you 4 messages $m_1$, $m_2$, $m_3$ and $m_4$.
        \item $A$ somehow decides which of these possible messages is $m$.
        %HUR REFERERA TILL ALGORITM?
    \end{enumerate}
    \end{description}
    
%   MERKLE-HELLMAN KNAPSACK
    \subsubsection{Merkle-Hellman knapsack encryption}
    is the first concrete realization of a public-key cryptographic system. It is based on the mathematical problem called the \textit{superincreasing subset sum problem}. It differs from RSA through the cryptation being only one-way. The public key is only used for encryption and the private key is only used for decryption. You can not get a message encrypted with the private key and decrypt it using the public key. Therefore it can't be used for cryptographic signing. Key generation uses the following steps.
    \begin{enumerate}
    \item An integer $n$ is fixed as the length of the public key.
    \item Choose a superincreasing sequence ($b_1$,$b_2$,...,$b_n$).
    \item Choose a integer $M$ such that $M$ > $b_1$+$b_2$+...+$b_n$.
    \item Choose another integer $W$ such that $1 \leq W \leq M - 1$ and gcd($W$,$M$) = 1.
    \item Select a random permutation $\pi$ of the integers \{1,2,...,$n$\}.
    \item Compute $a_i$ = $Wb_{\pi(i)} \bmod M$ for $i$ = 1,2,...,$n$.
    \item The public key is ($a_1$,$a_2$,...,$a_n$) and the private key is ($\pi$,$M$,$W$,($b_1$,$b_2$,...,$b_n$)).
    \end{enumerate}
    
    \begin{description}
    \item[Case] $B$ encrypts a message $m$ for $A$ which $A$ then decrypts.
    \item[Encryption] What $B$ should do.
    \begin{enumerate}
        \item Obtain $A$'s public key ($a_1$,$a_2$,...,$a_n$).
        \item Represent the message as binary string of length $n$, $m$ = $m_1m_2...m_n$.
        \item Compute the integer $c$= $m_1a_1$ + $m_2a_2$ + ... + $m_na_n$.
        \item Send $c$ to $A$.
    \end{enumerate}
    \item[Decryption] What $A$ should do.
    \begin{enumerate}
        \item Compute $d$ = $W^{-1}c \bmod M$.
        \item By solving a superincreasing subset sum problem using the algorithm, find integer $r_1$,$r_2$,...,$r_n$. $r_i \in \{0,1\}$, such that $d$ = $r_1b_1$ + $r_2b_2$ + ... + $r_nb_n$.
        \item The message bits are $m_i$ = $r_{\pi(i)}$,$i = 1,2,...,n$.
        \item $m$ = $m_1m_2...m_n$
        %HUR REFERERA TILL ALGORITM?
    \end{enumerate}
    \end{description}
%   ELGAMAL   
    \subsubsection{ElGamal public-key encryption}
    also known as the Diffie-Hellman key exchange. The security of ElGamal is based on the intractability of the discrete logarithm problem and the Diffie-Hellman problem, as \cite{handcrypt} wrote on p.294. Two versions of the ElGamal public-key encryption will be described, the basic and the generalized version. % NÄMN SKILLNAD?: The difference between them is that the generalized version uses Groups, which is ... % FÖRKLARA
    The key generation for the basic ElGamal is executed in the following steps:
    \begin{enumerate} % KEY GENERATION
        \item Generate a large random prime $p$ and a generator $\alpha$ of the multiplicative group $\mathbb{Z}^*_p$ of the integers modulo $p$ using algorithm 4.84.
        %HUR SKRIVA ATT VI ANVÄNDER ALGORITMEN
        \item Select a random integer $a$, $1 \leq a \leq p - 2$, and compute $\alpha^\alpha \bmod p$ using algorithm 2.143. % TODO
        \item Which gives the following: $A$`s public key is $(p,\alpha, \alpha^\alpha)$ and $A$`s private key is $\alpha$.
    \end{enumerate}
    \textbf{Basic ElGamal public-key encryption}
    The next part describes how the basic ElGamal encryption and decryption functions, in the case where $B$ is to encrypt a message $m$ for $A$, which is to be decrypted by $A$, according to p.295 in \cite{handcrypt}.
    \begin{description}
        \item[Encryption] In order to encrypt the message $m$, $B$ has to complete the following steps: 
        \begin{enumerate}
            \item Obtain $A$`s authentic public key ($p, \alpha, \alpha^\alpha$) 
            \item Represent the message as an integer $m$ in the range $\{0, 1, ..., p - 1 \}$
            \item Select a random integer $k$, $1 \leq k \leq p - 2$.
            \item Compute $\gamma = \alpha^k \bmod p$ and $\delta = m \dot (\alpha^\alpha)^k \bmod p$
            \item Send the cipher text $c = (\gamma, \delta)$ to $A$.
        \end{enumerate}
        \item[Decryption] To recover the plain text $m$ from the cipher text $c$, $A$ should proceed with the following steps:
        \begin{enumerate}
            \item Use the private key $a$ to compute $\gamma^{p-1-a} \bmod p$
            \item Retrieve $m$ by computing $(\gamma^{-a}) \cdot \delta \bmod p$. 
        \end{enumerate}
    \end{description}
    The generalized version of ElGamal encryption is based on the generalization of the basic version, i.e. the multiplicative group $\mathbb{Z}^*_p$ can be generalized to work in any finite cyclic group $G$, according to \cite{handcrypt} on p.297. The group $G$ should be carefully chosen to fulfill the following conditions:
    \begin{enumerate}
        \item for efficiency, the group operation in $G$ should be relatively easy to apply, and
        \item for security, the discrete logarithm problem in $G$ should be computationally infeasible.
    \end{enumerate}
    The key generation for the generalized version of ElGamal can be described in three steps, where each entity $A$ should perform each step, after obtaining a public key and a corresponding private key:
    \begin{enumerate} % KEY GENERATION
        \item Select an appropriate cyclic group $G$ of order $n$, with generator $\alpha$, assuming that $G$ is written multiplicative. 
        \item Select a random integer $a$, where $1 \leq a \leq n - 1$, and compute the group element $\alpha^\alpha$.
        \item $A$'s public key is $(\alpha, \alpha^\alpha)$, together with a description how to multiply the elements in $G$ and $A$'s private key is $a$.
    \end{enumerate}
    \textbf{Generalized ElGamal public-key encryption}
    $B$ encrypts a message $m$ for $A$, which $A$ decrypts.
    \begin{description} 
        \item[Encryption] $B$ should do the following:
        \begin{enumerate}
            \item Obtain $A$'s authentic public key $(\alpha, \alpha^\alpha)$.
            \item Represent the message as an element $m$ of the group $G$.
            \item Select a random integer $k$, $1 \leq k \leq n - 1$.
            \item Compute $\gamma = \alpha^k$ and $\delta = m \cdot (\alpha^\alpha)^k$.
            \item Send the cipher text $c = (\gamma, \delta)$ to $A$.
        \end{enumerate}
        \item[Decryption] To recover the plain text $m$ from $c$, $A$ should do the following: 
        \begin{enumerate}
            \item Use the private key $a$ to compute $\gamma^a$ and then compute $\gamma^{-a}$.
            \item Recover $m$ by computing $(\gamma^{-a}) \cdot \delta$.
        \end{enumerate}
    \end{description}
\subsubsection{McEliece public-key encryption}
The McEliece public-key encryption scheme is based on the
difficulty of decoding a general linear code, i.e. an error correcting code where any linear combination of code words, is also a codeword. \cite{code} The idea behind this scheme is to first select a particular code for which an efficient decoding algorithm is known, and then to disguise the code as a general linear code. Due to the fact that the procedure of decoding an arbitrary linear code is NP-hard, a description of the original code can serve as the private key and a description of the transformed code serves as the public key, as written on p.298 \cite{handcrypt}.

\subsubsection{Probabilistic public-key encryption}
Probabilistic encryption can be described as using randomness to achieve a provable and very strong level of security, according to p.306 \cite{handcrypt}. This is used to prevent the occurrence of equal cipher texts when encrypting the same plain text message several times.

\subsection{Authentication}
Public-key encryption sounds really great in theory and the math checks out. However there are of course issues with it. The biggest issue is authentication. If someone sends you their public key, how do you know that they are who they say they are? A man in the middle attack is an example of this problem.

\subsubsection{Man in the middle attack}
A man in the middle attack is an attack which lives on the problem of authentication.  An example would be that you are requesting the public key of Google with a message to their address. The attacker catches the message and sends back his public key instead of Google's. Now you will of course assume that you have Google's public key and can send encrypted messages to them. What can happen is that the attacker will just take your packets, read them, send them to Google, read their response and send them back to you. He can do all of this without leaving any traces and neither you nor Google will know that he could read your "encrypted" information. He can also pretend to be Google and read your messages and send responses in their name. To send secure encrypted information we need some sort of authentication to prove you are who you say you are.

%-------------------------------------------------------------
%   METHOD
%-------------------------------------------------------------
\section{Method} %Redo
In order to attain the knowledge to write this report, a literature study will be performed. The literature study will have the goal of finding sources of relevant information and it will be performed by searching for research papers online and on the recommended sources given on the course page. When material is found which appears to be relevant, it will be analyzed to find if it is legitimate to use in this report. 

%Initially we performed a literature study in order to find sources of relevant information for our report. We did this by searching online and in the recommended literature on the course homepage.  Once we found literature which appealed to us, we analyzed the material to find out if it was legitimate to use in our report. In the following weeks we proceeded by structuring our report in sections which were recommended on the course web page.


%-------------------------------------------------------------
%   RESULTS
%-------------------------------------------------------------
\section{Results}
Due to the fact that encryption transformation is public information, leads to the lack of data origin authentication in public-key encryption, where data origin authentication can be described as a mechanism where the data is authenticated and proven to legitimate. The next sections describe what available solutions exists. % TODO

\subsection{Certificate authorities}
In order to authenticate that a website is who it say it is,
the currently used solution is to use certificates. A certificate
normally contains a number of fields, such as: 
\begin{itemize}
    \item id,
    \item subject, 
    \item what signature algorithm has been used to create
    \item the signature,
    \item signature,
    \item issuer,
    \item valid-from,
    \item valid-to,
    \item what purpose the key was intended for,
    \item public key, fingerprint algorithm,
    \item fingerprint.
\end{itemize}
The creators and holders of these certificates are called certificate authorities (CA's).

\subsection{Bitcoin}
About bitcoin...
\subsection{Web of Trust}

%-------------------------------------------------------------
%   CONCLUSIONS
%-------------------------------------------------------------
\section{Conclusions}
%SKRIV FINA SLUTSATSER
[Awesome Text]

%-------------------------------------------------------------
%   DISCUSSION
%-------------------------------------------------------------
\section{Discussion}
%SKRIV FIN DISKUSSION
Idea: By the above description it is clear that using public-key encryption on its own, does not provide data origin authentication, due to the fact that $A$'s encryption transformation is public information. Data origin authentication can be described as a  mechanism where the data is authenticated. 

% The following two commands are all you need in the
% initial runs of your .tex file to
% produce the bibliography for the citations in your paper.
\bibliographystyle{abbrv}
\bibliography{sigproc}  % sigproc.bib is the name of the Bibliography in this case
% You must have a proper ".bib" file
%  and remember to run:
% latex bibtex latex latex
% to resolve all references
%
% ACM needs 'a single self-contained file'!
%
%APPENDICES are optional
%\balancecolumns

\end{document}
