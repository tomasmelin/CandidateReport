\documentclass{acm_proc_article-sp}
\usepackage[utf8]{inputenc}

\hyphenation{authentication}
\hyphenation{cryptography}
\hyphenation{symmetric}
\hyphenation{asymmetric}
\hyphenation{vulnerable}
\begin{document}

%Vilka hot finns mot krypteringsmetoder som använder sig av asymmetrisk kryptering?
\title{What vulnerbilities and threats exists for public-key cryptography?}
\numberofauthors{2} 
\author{
\alignauthor
Tomas Melin\\
        \affaddr{Linköping University}\\
        \affaddr{Linköping, Sweden}\\
        \email{tomme578@student.liu.se}
% 2nd. author
\alignauthor
Tomas Vidhall\\
        \affaddr{Linköping University}\\
        \affaddr{Linköping, Sweden}\\
        \email{tomvi780@student.liu.se}
}
\date{17 March 2014}

\maketitle
\begin{abstract}
%% SKRIV ABSTRAKT
[Awesome Text]
\end{abstract}

\keywords{ACM proceedings, \LaTeX, text tagging} % NOT required for Proceedings

\section{Introduction}
%Skriv inledning här!

\subsection{Motivation}
%SKRIV MOTIVATION HÄR
[Awesome Text]

\subsection{Purpose}
%SKRIV SYFTE
[Awesome Text]

\subsection{Questions of Interest}

\begin{itemize}
    \item %Hur fungerar asymmetrisk kryptering? 
    How does asymmetric cryptography work?
    \item %Hur har asymmetrisk kryptering historiskt sett förändrats då olika implementationer har knäckts?
    How has asymmetric cryptography implementations changed through history?
    \begin{itemize}
        \item Has it been affected by cracked implementations?
    \end{itemize}
    
    \item %Vilka sorts attacker är asymmetriska krypteringsmetoder sårbar mot? 
    What attacks are asymetric cryptographys vulnerable to?
    \begin{itemize} 
        \item %Vad kan man göra som ‘Man in the Middle’?
        What weaknesses can a 'Man in the Middle' take advantage of?
        \item %Kan man motverka dessa attacker med autentisering?
        Is it possible to prevent these attacks through authentication?
    \end{itemize}
    
    \item %Hur fungerar Diffie-Hellmans metod för nyckelutbyte?
    How does the Diffie-Hellman key exchange work?   
    \begin{itemize}
        \item %Vilka sorters attacker är Diffie-Hellman sårbar mot?
        What attacks is Diffie-Hellman vulnerable to?
        \item %Varför vill man kryptera symmetriskt när man redan kan kommunicera med asymmetrisk kryptering? 
        What is the benefit of using symmetric cryptography compared to asymetric cryptography?
        \item Why do we use Diffie-Hellman to convert our asymmetric crypto to a symmetric one?
    \end{itemize}
    
\end{itemize}

\subsection{Limitations}
%SKRIV AVGRÄNSNINGAR HÄR
[Awesome Text]

\section{Background}
%SKRIV BAKGRUND HÄR (VID BEHOV) 
[Awesome Text]

\section{Theory}
%SKRIV VAD VI HAR LÄST
[Awesome Text]

\section{Method}
%SKRIV METOD
[Awesome Text]

\section{Results}
%SKRIV RESULTAT
[Awesome Text]

\section{Discussion}
%SKRIV FIN DISKUSSION
[Awesome Text]

\section{Conclusions}
%SKRIV FINA SLUTSATSER

\section{Bibliography}

\subsection{Citations}


% The following two commands are all you need in the
% initial runs of your .tex file to
% produce the bibliography for the citations in your paper.
\bibliographystyle{abbrv}
%\bibliography{sigproc}  % sigproc.bib is the name of the Bibliography in this case
% You must have a proper ".bib" file
%  and remember to run:
% latex bibtex latex latex
% to resolve all references
%
% ACM needs 'a single self-contained file'!
%
%APPENDICES are optional
%\balancecolumns

%\balancecolumns
% That's all folks!
\end{document}
